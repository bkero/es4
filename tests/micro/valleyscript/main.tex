\documentclass{article}
\usepackage{amssymb}
%\usepackage{gastex}
\usepackage{graphics}
% ---------------------------------------------------------------------	%
% Horizontal brackets.							%
% ---------------------------------------------------------------------	%

\newcommand{\hbra}{
\hbox to \textwidth{\vrule width0.3mm height 1.8mm depth-0.3mm
                    \leaders\hrule height1.8mm depth-1.5mm\hfill
                    \vrule width0.3mm height 1.8mm depth-0.3mm}}
\newcommand{\hket}{
\hbox to \textwidth{\vrule width0.3mm height1.5mm
                    \leaders\hrule height0.3mm\hfill
                    \vrule width0.3mm height1.5mm}}


\renewcommand{\hbra}{
\hbox to \columnwidth{\vrule width0.3mm height 1.8mm depth-0.3mm
                    \leaders\hrule height1.8mm depth-1.5mm\hfill
                    \vrule width0.3mm height 1.8mm depth-0.3mm}}
\renewcommand{\hket}{
\hbox to \columnwidth{\vrule width0.3mm height1.5mm
                    \leaders\hrule height0.3mm\hfill
                    \vrule width0.3mm height1.5mm}}


%\newcounter{figure}

\newenvironment{cfproof}%
  {%\begin{quote}
    %\item[]
%   \begin{quote}
     \normalsize{\sc Proof:}\hspace*{0mm}
}
  {\proofcomplete
%\end{quote}%
\smallskip 
   }


\newcommand{\proofcomplete}{ %\hspace*{\fill}
$\square$
}


\newenvironment{display}[1]{
  \refstepcounter{figure}
 % \begin{tabbing}
 % \hspace{1.5em} \= \hspace{.40\linewidth - 2em} \= \hspace{2em} \= \kill
%  \medskip
\noindent
  \textbf{Figure~\thefigure:  #1}\\
  \hbra\\[-.8ex]
  \nopagebreak
  %\small
 }{ 
 \\[-2ex]
 \nopagebreakh
  \hket
  \nopagebreak
% \medskip
 % \end{tabbing}
 }


\newcommand{\htbra}{
\hbox to \textwidth{\vrule width0.3mm height 1.8mm depth-0.3mm
                    \leaders\hrule height1.8mm depth-1.5mm\hfill
                    \vrule width0.3mm height 1.8mm depth-0.3mm}}
\newcommand{\htket}{
\hbox to \textwidth{\vrule width0.3mm height1.5mm
                    \leaders\hrule height0.3mm\hfill
                    \vrule width0.3mm height1.5mm}}


\newenvironment{displayfigure}[2]{
  \begin{figure}[#1]
  \refstepcounter{figure}
  \medskip
  \textbf{Figure~\thefigure:  #2}\\
  \hbra\\[-.8ex]
  \nopagebreak
  %\small
 }{ 
  \\[-.8ex]
 \nopagebreak
  \hket
  \nopagebreak
 \medskip
  \end{figure}
 % \end{tabbing}
 }

\newenvironment{displayfigure*}[1]{
  \begin{figure*}[tph!]
   \refstepcounter{figure}
 \medskip
  \textbf{Figure~\thefigure:  #1}\\
  \htbra\\[-.8ex]
  \nopagebreak
  \footnotesize
 }{ 
  \\[-.8ex]
 \nopagebreak
  \htket
  \nopagebreak
 %\medskip
  \end{figure*}
 % \end{tabbing}
 }

\newenvironment{displayname}[1]{
  \refstepcounter{figure}
  \begin{tabbing}
  \hspace{1.5em} \= \hspace{.40\linewidth - 2em} \= \hspace{2em} \= \kill
  \textbf{Figure~\thefigure: #1}\\[-.8ex]
  \hbra\\[-.8ex]
  \nopagebreak
 }{
  \\[-.8ex]\hket

  \end{tabbing}
 }

%%%%%%%% typesetting type rules %%%%%%%%%%%%%
\newenvironment{trules}
{\[\begin{array}{@{}c@{}r@{}}}
{\end{array}\]}
\newcommand{\headtrule}[2]{
	\multicolumn{2}{@{}l@{}}{\mbox{#2}} \hfil \fbox{#1}\\
	\multicolumn{2}{@{}l@{}}{\hskip3.25in}\\
}
\newcommand{\trule}[3]{
\frac 
	{\strut\begin{array}{@{}c@{}} #2 \end{array}} 
	{\strut\begin{array}{@{}c@{}} #3 \end{array}}
&\rel{#1} \\~\\
}
\newcommand{\rtrule}[3]{  % restatement in proof
Suppose $#3$ via the rule:
\[
\frac 
	{\strut\begin{array}{@{}c@{}} #2 \end{array}} 
	{\strut\begin{array}{@{}c@{}} #3 \end{array}}
\qquad\rel{#1}
\]
}
\newcommand{\trulespc}[4]{
\frac 
	{\strut\begin{array}{@{}c@{}} #2 \end{array}} 
	{\strut\begin{array}{@{}c@{}} #3 \end{array}}
#4
&\rel{#1} \\~\\
}
\newcommand{\trulenoname}[2]{
\frac 
	{\strut\begin{array}{@{}c@{}} #1 \end{array}} 
	{\strut\begin{array}{@{}c@{}} #2 \end{array}}
}
\newcommand{\nrln}[3]{
\mbox{
$\begin{array}{l}
\rel{#1} \\
\quad\frac 
	{\strut\begin{array}{c} #2 \end{array}} 
	{\strut\begin{array}{c} #3 \end{array}}
\end{array}
$}}
\newcommand{\rel}[1]{\raisebox{0ex}[0ex][0ex]{\small [\textsc{#1}]}}
\newcommand{\relrel}[1]{\scriptsize [\textsc{#1}]}
\newcommand{\mydefhead}[2]{\multicolumn{2}{l}{{#1}}&\mbox{\emph{#2}}\\}
\newcommand{\mydefcase}[2]{\qquad& #1 &\mbox{#2}\\}
\newcommand{\headrule}[2]{\multicolumn{3}{@{}l}{\mbox{#2}} & \fbox{#1}\\}



\newcounter{secthm}

%not-restate-able
\newtheorem{assume}[secthm]{Assumption}
\newtheorem{defn}[secthm]{Definition}
\newtheorem{coro}[secthm]{Corollary}

% restate-able
\newtheorem{lemma}[secthm]{Lemma}
\newtheorem{theorem}[secthm]{Theorem}

\newcommand{\rntheorem}[3]{
    \begin{theorem}[#2 foo]
    \label{thm:#1}
    #3
    \end{theorem}
    \expandafter\def\csname restate#1\endcsname{ {
%        \trivlist \item[\hskip \labelsep{\bfseries Restatement of Theorem\ \ref{thm:#1} (#2)}]\itshape
        \trivlist \item[\hskip \labelsep{\sc Restatement of Theorem\ \ref{thm:#1} (#2)}]\itshape
        #3
        \endtrivlist }
     }
 }


\newcommand{\rtheorem}[2]{
    \begin{theorem}
    \label{thm:#1}
    #2
    \end{theorem}
    \expandafter\def\csname restate#1\endcsname{ {
%        \trivlist \item[\hskip \labelsep{\bfseries Restatement of Theorem\ \ref{thm:#1}}]\itshape
        \trivlist \item[\hskip \labelsep{\sc Restatement of Theorem\ \ref{thm:#1}}]\itshape
        #2
        \endtrivlist }
     }
  }


\newcommand{\rnlemma}[3]{
    \begin{lemma}[#2]
    \label{lem:#1}
    #3
    \end{lemma}
    \expandafter\def\csname restate#1\endcsname{ {
%        \trivlist \item[\hskip \labelsep{\bfseries Restatement of Lemma\ \ref{lem:#1} (#2)}]\itshape
        \trivlist \item[\hskip \labelsep{{\sc Restatement of Lemma\ \ref{lem:#1}} (#2)}]\itshape
        #3
        \endtrivlist }
     }
  }


\newcommand{\rlemma}[2]{
    \begin{lemma}
    \label{lem:#1}
    #2
    \end{lemma}
    \expandafter\def\csname restate#1\endcsname{ {
%        \trivlist \item[\hskip \labelsep{\bfseries Restatement of Lemma\ \ref{lem:#1}}]\itshape
        \trivlist \item[\hskip \labelsep{\sc Restatement of Lemma\ \ref{lem:#1}}]\itshape
        #2
        \endtrivlist }
     }
  }


\setlength{\textwidth}{6.5in}
\setlength{\oddsidemargin}{0in}
\setlength{\textheight}{8.5in}
\setlength{\topmargin}{0in}
\setlength{\topskip}{0in}

\newcommand{\fun}[1]{\mbox{\it #1\/}}
\newcommand{\eg}{\emph{e.g.}}
\newcommand{\ie}{\emph{i.e.}}
\newcommand{\meaningf}[1]{\ensuremath{[\![ #1 ]\!]}}
\newcommand{\judge}[2]{\ensuremath{\Gamma\vdash{#1}:{#2}}}
\newcommand{\setc}[2]{ \{ #1 ~|~ #2 \}}
\newcommand{\setz}[1]{ \{ #1 \}}
\newcommand{\nt}[1]{\ba#1\ea}
\newcommand{\comment}[1]{}
\newcommand{\fs}[1]{\mbox{\it #1}}

%%%%%%%% language %%%%%%%%%%%%%%%

% deprecated
\newcommand{\mkref}[2]{\t{ref}~{#1}~#2}
\newcommand{\mkrefl}[3]{\t{ref}^{#3}~{#1}~#2}
\newcommand{\deref}[1]{!{#1}}
\newcommand{\dereft}[2]{!{#1}:{#2}}
\newcommand{\derefl}[2]{!^{#2}{#1}}
\newcommand{\assign}[2]{{#1}:={#2}}
\newcommand{\assignl}[3]{{#1}:=^{#3}{#2}}
\newcommand{\Ref}[1]{\t{Ref}~{#1}}

\newcommand{\slam}[3]{\lambda#1\!:\!#2.\,\,#3}
\newcommand{\lam}[4]{\lambda#1\!:\!#2.\,\,#4:#3}
\newcommand{\lete}[4]{\t{let}~{#1}:{#2}={#3}~\t{in}~{#4}} 
\newcommand{\lamt}[2]{#1\rightarrow #2}
\newcommand{\appl}[3]{(\app{#2}{#3})^{#1}}
\newcommand{\appt}[3]{(\app{#2}{#3}):{#1}}
\newcommand{\app}[2]{#1~#2}
\newcommand{\deltaf}[2]{\meaningf{#1}(#2)}
\newcommand{\Nat}{\t{N}}
\newcommand{\Int}{\t{Int}}
\newcommand{\Bool}{\t{Bool}}
\newcommand{\dynamic}{\t{*}} 
\newcommand{\true}{\t{true}}
\newcommand{\false}{\t{false}}
\newcommand{\hole}{\bullet}
\newcommand{\Lam}[3]{\Lambda #1.\,#2:#3}
\newcommand{\App}[2]{#1[#2]}
\newcommand{\Forall}[2]{\forall #1.\,#2}

\newcommand{\letexp}[3]{\t{let}~{#1}={#2}~\t{in}~{#3}}
\newcommand{\lett}[4]{\t{letXXX}~{#1}:{#2}={#3}~\t{in}~{#4}}
\newcommand{\cast}[2]{#1~\t{as}~#2}
\newcommand{\arrowt}[2]{#1\rightarrow #2}
\newcommand{\defeq}{\stackrel{\mathrm{def}}{=}}

\newcommand{\subtypeword}[1]{\,\,<\!\!\!{\tiny{#1}}\,\,}
\renewcommand{\subtypeword}[1]{\,\,<\!\!_{\tiny{#1}}\,\,}
\newcommand{\subtyped}[3]{#2 \subtypeword{#1} #3}
\newcommand{\subtype}[2]{#1 \subtypeword{\top} #2}
%\newcommand{\subtypem}[2]{#1 \subtypeword{\bot} #2}
\newcommand{\notd}[1]{\overline #1}

\newcommand{\convert}[2]{{#1}\convertword{#2}}
\newcommand{\convertword}{\mbox{\,\,$\sim :$\,\,}}

\newcommand{\implies}[2]{#1\Implies #2}

\renewcommand{\t}[1]{{\tt #1}}

%%%%%%%%%%% math %%%%%%%%%%%%%%%%%%%%%%%%%%%%%%
\newcommand{\Or}{\vee}
\newcommand{\meet}{\sqcap}
\newcommand{\join}{\sqcup}
\newcommand{\And}[0]{\wedge}
\newcommand{\Implies}[0]{\Rightarrow}
\newcommand{\Iff}[0]{\Leftrightarrow}

%======= judgments ===============
\renewcommand{\judge}[3]{#1\vdash #2\,:\,#3}
\newcommand{\judgeE}[1]{\vdash #1}
\newcommand{\judgeT}[2]{#1\vdash #2}
\newcommand{\red}[0]{\longrightarrow}  % reduction
\newcommand{\lred}[0]{\red}  % local reduction
\newcommand{\judgers}[3]{#2 \rightarrow^{#1} #3}
\newcommand{\compilesymbol}{\hookrightarrow}
\newcommand{\judgec}[4]{#1 \vdash #2 \,\compilesymbol\, #3 \,:\, #4 }

\newcommand{\judges}[3]{#1\vdash\!\!\!\vdash #2\,:\,#3}
 
%=== objects
\newcommand{\objty}[1]{\{#1\}}
\newcommand{\obje}[2]{\{#1\}:{#2}}
\newcommand{\objv}[1]{\{#1\}}
\newcommand{\objget}[2]{#1.#2}
\newcommand{\objcall}[2]{#1.#2()}
\newcommand{\objset}[3]{#1.#2:=#3} 

	
\newcommand{\wrapty}[1]{\t{wrap}~#1}
\newcommand{\likety}[1]{\t{like}~#1}
\newcommand{\wrapv}[2]{#1~\t{wrapped}~#2}
%\renewcommand{\wrapv}[2]{\wrap{#1}{#2}}
 
\newcommand{\wrap}[2]{#1~\t{wrap}~#2}
%\newcommand{\assignable}[2]{#1<#2}
\newcommand{\likev}[2]{#1 \sqsubset_{\sigma} #2 }
\renewcommand{\likev}[2]{#1 ~\fun{XXXX like}_{\sigma}~ #2 }
\newcommand{\compatible}[2]{#1 XXX \sim: #2 }
\newcommand{\convertible}[2]{#1 XXX \prec #2 }
 
%\newcommand{\allocty}[2]{#1:_{\sigma}#2}
\newcommand{\allocty}[1]{ty(#1)}
\newcommand{\curty}[1]{cty_{\sigma}(#1)}
\renewcommand{\convert}[2]{\fun{convert}_{\sigma}(#1,#2)}
\newcommand{\istype}[2]{#1~\t{is}_{\sigma}~ #2}
\newcommand{\safe}[1]{\t{safe}~#1}
 

%%%%%%%%%%%%%%%%%%%%%%%%%%%%%%%%%%%%%%%%%%%%%%%%%%%%%%%%%%%%%%%%%%%%%%%%%%				   

\begin{document}


\title{ValleyScript: It's Like Static Typing
}
\author{Cormac Flanagan}


%\date{\today}
\maketitle

\begin{abstract}
We formalize the ES4 notions of \t{like} types and \t{wrap} operators for a lambda-calculus with ES4-style objects,
to better understand these concepts and to clarify what guarantees can be provided by the verifier in strict mode.
\end{abstract}

%%%%%%%%%%%%%%%%%%%%%%%%%%%%%%%%%%%%%%%%%%%%%%%%%%%%%%%%%%
\section{Change Log}
\begin{itemize}
\item
6 Oct: extended with generic function, as a warm-up for implementing all this in the verifier.
\item
8 Nov: extended with conversion from $\Int$ to $\Bool$.
\item
14 Nov: extended with generic functions.
\item
14 Nov: added wrap as a type constructor.
\item 
20 Nov: rework based on allocated and current types
\item
26 Nov: rework with two subtype relations
\item
todo: add typedefs, rework verifier and optimizer
\end{itemize}

\clearpage
\section{Language Overview}

We consider the implementation of a \emph{gradual typed} language that supports both
typed and untyped terms, which interoperate in a flexible manner.
We begin by defining the syntax of terms and types in the language: see Figure~\ref{fig:syntax}.
In addition to the usual terms of the lambda calculus (variables, abstractions, and application), 
the language also includes constants and expressions to create, dereference, and update objects.
It also includes \t{is} expressions, which check that a value has a particular type,
and \t{wrap} expressions, which, if necessary, wrap the given value to ensure that it behaves like it has that type. 

The type language is fairly rich. In addition to  base types ($\Int$ and $\Bool$), function types,
and object types, the language includes additional types related to gradual typing.
The type $\dynamic$ is (roughly) a top type, and indicates that no static type information is known.
%The type $\dynobjty$ denotes an object type, where no static type information
%is known about the names of fields of that object.
The type $\likety{T}$ describes values whose value components match $T$, but whose type components may be more vague than $T$, due to the presence of the type $\dynamic$. (Due to imperative constructs, that matching-value guarantee does not persist, and so \t{like} types are helpful for debugging but do not provide strong guarantees.)

We include generic function definition, generic function application, and the associated polymorphic types and type variables.

%Any value of type $\likety{T}$ can be assigned to a variable to type $\wrapty{T}$,
%but that value is immediate wrapped (if necessary) so that the resulting value is guaranteed to have type $T$. 


\begin{displayfigure}{th}{\label{fig:syntax}Syntax}
\[
\begin{array}{llr}
	\mydefhead{e ::=\qquad\qquad\qquad\qquad\qquad}{Terms:} 
	\mydefcase{n								}{integer constants} 
	\mydefcase{x								}{variable} 
	\mydefcase{\lam{x}{S}{T}{e} 				}{abstraction} 
	\mydefcase{\app{e}{e} 					}{application} 
	\mydefcase{\obje{\bar{l}=\bar{e}}{T}		}{object expression}
	\mydefcase{\objget{e}{l}					}{member selection}
	\mydefcase{\objset{e}{l}{e}				}{member update}
	\mydefcase{\cast{e}{T}					}{runtime type check}
\\
	\mydefhead{S,T::=}{Types:} 
	\mydefcase{\Int 							}{integers}
%	\mydefcase{\Bool							}{booleans}
	\mydefcase{\lamt{S}{T} 					}{function type}
	\mydefcase{\objty{\bar{l}:\bar{T}}	   	}{object types}
	\mydefcase{\dynamic 		    				}{dynamic type}
\end{array}
\]
\end{displayfigure}

%------------------------------------------

%%%%%%%%%%%%%%%%%%%%%%%%%%%%%%%%%%%%%%%%%%%%%%%%%%%%%%%%%%
\clearpage
\section{Subtyping}


\begin{displayfigure}{th}{Subtyping}
\label{fig:subtype}
\begin{trules}
\headtrule{$\subtyped{m}{S}{T}$}{Subtyping}
\trule{Sub-Refl}{
	}{
		\subtyped{m}{T}{T}
	}
\trule{Sub-Dyn-Top}{
	}{
		\subtyped{\top}{T}{\dynamic}
	}
\trule{Sub-Dyn-Bot}{
	}{
		\subtyped{\bot}{\dynamic}T
	}
\trule{Sub-Arrow}{
		%(\subtyped{m}{T_1}{S_1}\mbox{ or }T_1=\dynamic)\\
		\subtyped{\notd m}{T_1}{S_1} \qquad
		\subtyped{m}{S_2}{T_2}
	}{
		\subtyped{m}{(\lamt{S_1}{S_2})}{(\lamt{T_1}{T_2})}
	}
\trule{Sub-Obj}{
		\subtyped{\notd m}{T_i}{S_i}\qquad
		\subtyped{m}{S_i}{T_i} \qquad\mbox{for $i\in 1..n$}
	}{
		\subtyped{m}{\objty{l_i:S_i^{i\in 1..n+m}}}
				{\objty{l_i:T_i^{i\in 1..n}}}
	}
\end{trules}
\end{displayfigure}

The type system supports the type $\dynamic$, which denotes an absence of static type information. We would like $\lamt\dynamic\dynamic$ to be the most general function type.
That is, if a variable $f$ has type $\lamt{\dynamic}{\dynamic}$, then no static type information is available about the true domain of the function bound (at run-time) to $f$; instead, a dynamic type check is performed at run-time. 

Since $\lamt\dynamic\dynamic$ is the maximal function type, we need that:
\[
	(\lamt{S}{T})  ~~<:~~  (\lamt\dynamic\dynamic)
\]
for any function type $\lamt{S}{T}$. However,  the usual rule for function subtyping then implies that $T~<:~\dynamic$ (that is, $\dynamic$ is a maximal or top type), and also that
$\dynamic ~<:~ S$ (that is, $\dynamic$ is a minimal or bottom type). 

To resolve these two differing interpretations of $\dynamic$, we introduce \emph{two} subtype relations, $\subtypeword{\top}$ (the ''top-subtype'' or simply ''subtype'' relation) and $\subtypeword{\bot}$ (the ''bottom-subtype'' relation), that differ in their interpretation of $\dynamic$. 
Figure~\ref{fig:subtype} defines these relations, using the meta-variable $m$ to range over $\setz{\top,\bot}$.
The rule \rel{Sub-Arrow} contravariant in the domain position, and also switches the interpretation of $\dynamic$. The rule \rel{Sub-Obj} introduces width-subtyping on objects, and again switches the interpretation of $\dynamic$ for the contravariant check. (Note that writes to an object field in general require a run-time check, analogous to the covariant array write check in Java.)

Unlike the consistency relation of earlier work~\cite{SiekTaha}, both subtyping relations are transitively closed.

\begin{lemma}
$\subtypeword{m}$ is reflexively and transitively closed.
\end{lemma}
TBP. FIXME.

We have
\[
\begin{array}{rcl}
	(\lamt{S}{T})  		&\subtypeword{\top}&  (\lamt\dynamic\dynamic) \\
	\objty{l:T,\dots}	&\subtypeword{\top}&  \objty{l:\dynamic} \\
	\objty{\dots}	   	&\subtypeword{\top}&  \objty{} \\
\end{array}
\]

%%%%%%%%%%%%%%%%%%%%%%%%%%%%%%%%%%%%%%%%%%%%%%%%%%%%%%%%%%
\clearpage
\section{Evaluation}

We next describe the evaluation semantics of the language. 
The set of values in the language is given by:
\[
\begin{array}{llr}
	\mydefhead{v ::=\qquad\qquad\qquad\qquad\qquad}{Values:} 
	\mydefcase{n								}{integer constant} 
	\mydefcase{\lam{x}{S}{T}{e} 				}{abstraction} 
	\mydefcase{a_T							}{object address with type $T$}
	\\
	\mydefhead{o ::=\qquad\qquad\qquad\qquad\qquad}{Object value:} 
	\mydefcase{\obje{\bar l = \bar v}{T}		}{object value}
\end{array}
\]
Each object address $a$ is annotated with the type $T$ of the object it points to.
A \emph{object store} $\sigma$ maps object addresses $a_T$ to object values of the form $\objv{\bar{l}=\bar v}$.
Values and object values in the store are \emph{closed} in that they do not contain free program variables $x$;
though they may contain object addresses.

Every value has an \emph{allocated type} according to the function $\allocty{v}$:
\[
\begin{array}{rcll}
		\allocty{n}&=&\Int \\
		\allocty{b}&=&\Bool\\
		\allocty{\lam{x}{S}{T}{e}}&=&(\lamt{S}{T})\\
		\allocty{a_T}&=&T   
\end{array}
\]
The allocated type of an object is invariant and independent of the store $\sigma$. 
A \emph{type tag} is a type that can be returned by $\allocty{v}$; it includes $\Int$, function, and object types, but excludes $\dynamic$. 
We use the notation $T.l$ to denote $S$ if $T=\objty{l:S,\dots}$, and to denote $\dynamic$ if $T$ is a different object type.

An evaluation context is:
\[
C ~~::=
\begin{array}[t]{@{}l}
		~~~\hole
%	~|~ 	\lam{x}S{C}
	~|~	\app{C}{t} 
	~|~	\app{v}{C}
%	~|~	\App{C}{T} 
	~|~ \cast{C}{T} 
%	~|~ \wrap{C}{T} \\
	~|~ \obje{\bar l=\bar v,l=C, \bar l =\bar e}{T}
	~|~ \objget{C}{l}
%	~|~ \objcall{C}{l}
	~|~ \objset{C}{l}{e}
	~|~ \objset{v}{l}{C} 
\end{array}
\]
A \emph{state} is a pair of an object store and a current expression.
The evaluation relation on states is defined by the rules in Figure~\ref{fig:eval}.

Several rules refer to the function $\convert{v}{T}$, which checks if the value $v$ can be converted to the type $T$: 
see  Figure ~\ref{fig:convert}.  

\begin{displayfigure}{th}{Operational Semantics}
\label{fig:eval} 
\label{fig:convert} 
\footnotesize
{\bf Evaluation Rules}
\[
\begin{array}{@{}rclll}
	\sigma,C[{\app{(\lam{x}{S}{T}{t})}{v}}]
	&\lred&
	\sigma,C[\cast{t[x:=v']}{T}]
	&\mbox{if $v'=\convert v S$}
	&\rel{E-Beta}
\\~\\
	\sigma,C[\cast{v}{T}]
	&\lred&
	\sigma,C[v']
	&\mbox{if $v'=\convert v T$}
	&\rel{E-As}
\\~\\
	\sigma,C[\obje{l_i=v_i^{i\in 1..n}}{T}]
	&\lred&
	\sigma[a_T:=(\objv{l_i=v'_i})],C[a_T] 
	&\mbox{if $v_i'=\convert{v_i}{T.l_i}$, $a_T$ fresh, $T=\objty{\dots}$}
	&\rel{E-Alloc} 
	%&&&\mbox{$T=\objty{l_i:T_i^{i\in 1..n}}$, }
\\~\\
	\sigma,C[\objget{a_T}{l}]
	&\lred&
	\sigma,C[v]
	&\mbox{if $\sigma(a_T)=\objv{l=v,\dots}$}
	&\rel{E-Get}
\\~\\
	\sigma,C[\objset{a_T}{l}{v}]
	&\lred&
	\sigma[a_T,l:=v'],C[v] 
	&\mbox{if $v'=\convert v {T.l}$}
	&\rel{E-Assign} 
\\~\\
\end{array}
\]
{\bf Dynamic Type Checks}
\[
\begin{array}{rcl}
	\convert{v}{T} = v &\mbox{iff}&\mbox{$\istype v T$} \\
	\istype{v}{T}      &\mbox{iff}& \mbox{$\subtype{\allocty v}{T}$} \\
\end{array}
\]
 \end{displayfigure} 


%%%%%%%%%%%%%%%%%%%%%%%%%%%%%%%%%%%%%%%%%%%%%%%%%%%%%%%%%%%%%%%%%%%%%%%
\clearpage
\section{Check Optimization}

We now sketch a 
type-based analysis that statically identifies dynamic type checks that can be eliminated.
We introduce the ``\safe{e}'' expression form to indicate where run-time checks are unnecessary.
\[
	e ::= \dots ~|~ \safe{e}
\]
It is straightforward to formulate the operational semantics of the extended language.

Figure~\ref{fig:typerulesopt} presents \emph{optimization rules}, which verify that $\safe \dots$ occurs in correct places;
it is straightforward to reformulate the analysis to infer these $\safe{}$ annotations.
These rules rely on the relation $S\Implies T$, which is designed to satisfy the lemma:

\begin{lemma}
If $\istype v{S}$ and $S\Implies T$ then $\istype v{T}$.
\end{lemma}

\begin{lemma}
The $\Implies$ relation is transitive. 
\end{lemma}
PROVEME

\begin{lemma}[No Failure]
For any term $e$ with no $\safe{}$ annotations, there exists $T$ such that
$\judges{\emptyset}{e}{T}$ .
\end{lemma}
PROVEME

\begin{lemma}[Soundness]
If $\judges{\emptyset}{e}{T}$, then $\safe{}$ operation in $e$ can never get stuck.
\end{lemma}
PROVEME


\begin{displayfigure}{th}{Implication on Types}
\label{fig:subtype}
\begin{trules}
\headtrule{$\implies{S}{T}$}{Implication on Types}
\trule{Imp-Refl}{
	}{
		\implies{T}{T}
	}
\trule{Imp-Dyn-Top}{
	}{
		\implies{T}{\dynamic}
	}
\trule{Imp-Arrow}{
		%(\implies{T_1}{S_1}\mbox{ or }T_1=\dynamic)\\
		\implies{T_1}{S_1} \qquad
		\implies{S_2}{T_2}
	}{
		\implies{(\lamt{S_1}{S_2})}{(\lamt{T_1}{T_2})}
	}
\trule{Imp-Obj}{
		\implies{T_i}{S_i}\qquad
		\implies{S_i}{T_i} \qquad\mbox{for $i\in 1..n$}
	}{
		\implies{\objty{l_i:S_i^{i\in 1..n+m}}}
				{\objty{l_i:T_i^{i\in 1..n}}}
	}
\end{trules}
\end{displayfigure}


\newcommand{\maybesafe}[1]{\t{safe}^?~#1}

\begin{displayfigure}{th}{Type Rules for Optimization}
\label{fig:typerulesopt}
\footnotesize
\begin{trules}
\headtrule{$\judges{E}{t}{T}$}{\underline{T}ype rules\qquad\qquad\qquad\qquad\qquad\qquad\qquad\qquad\qquad\qquad\qquad\qquad\qquad\mbox{}}
\trule{O-Var-Safe}{
		(x:T)\in E
	}{
		\judges{E}{\safe x}{T}
	}
\trule{O-Var-Unsafe}{
	}{
		\judges{E}{x}{\dynamic}
	}
\trule{O-Const}{
	}{
		\judges{E}{n}{Int}
	}
\trule{O-Fun-Safe}{
		\judges{E,x:S}{e}{T'} \qquad
		{T'} \Implies {T} 
	}{
		\judges{E}{\safe (\lam{x}{S}{T}{e})}{(\lamt{S}{T})}
	}
\trule{O-Fun-Unsafe}{
		\judges{E,x:S}{e}{T'}  
	}{
		\judges{E}{(\lam{x}{S}{T}{e})}{(\lamt{S}{T})}
	}
\trule{O-App-Safe}{
		\judges{E}{t_1}{(\lamt{S}{T})}\qquad
		\judges{E}{t_2}{S'} \qquad
		{S'}\Implies{S} \\
	%	(k,k')\in\setz{(\likety{},\likety{\!\!}),(\wrapty{},\epsilon),(\epsilon,\epsilon)} \\
	}{
		\judges{E}{\safe (\app{t_1}{t_2})}{T}
	}
\trule{O-App-Unsafe}{
		\judges{E}{t_1}{S}\qquad
		\judges{E}{t_2}{S'}  
	}{
		\judges{E}{(\app{t_1}{t_2})}{\dynamic}
	}
\trule{O-As}{
		\judges{E}{t}{S}%\qquad 
		%\compatible{S}{T}
	}{
		\judges{E}{\cast{t}{T}}{T}
	}
\trule{O-Alloc-Safe}{
		T=\objty{l_i:T_i^{i\in 1..n}} \qquad
		\judges{E}{t_i}{S_i} \qquad
		{S_i}\Implies{T_i}
	}{
		\judges{E}{\safe{(\obje{l_i=t_i^{i\in 1..n}}{T})}}{T}
	}
\trule{O-Alloc-Unsafe}{
		\judges{E}{t_i}{S_i} 
	}{
		\judges{E}{(\obje{l_i=t_i^{i\in 1..n}}{T})}{T}
	}
\trule{O-Get-Safe}{
		\judges{E}{e}{\objty{l:T,\dots}}  
		%k\in\setz{\likety{},\wrapty{},\epsilon}
	}{
		\judges{E}{\safe{\objget{e}{l}}}{T}
	}
\trule{O-Get-Unsafe}{
		\judges{E}{e}{S}
	}{
		\judges{E}{\objget{e}{l}}{\dynamic}
	}
\trule{O-Set-Safe}{
		\judges{E}{e_1}{\objty{l:T,\dots}} \qquad
		\judges{E}{e_2}{S} \qquad
		{S}\Implies{T}  
	}{
		\judges{E}{\safe{(\objset{e_1}{l}{e_2})}}{S}
	}
\trule{O-Set-Unsafe}{
		\judges{E}{e_1}{\dynamic} \qquad
		\judges{E}{e_2}{S}
	}{
		\judges{E}{\objset{e_1}{l}{e_2}}{S}
	}
\end{trules}
\end{displayfigure}

%------------------------------------------



%%%%%%%%%%%%%%%%%%%%%%%%%%%%%%%%%%%%%%%%%%%%%%%%%%%%%%%%%%%%%%%%%%%%%%%
\clearpage
\section{Strict Mode Type System}



\begin{displayfigure}{th}{Type Relations}
\begin{trules}
\headtrule{$\convertible{S}{T}$}{Convertible Types}
\trule{Con-Sub}{
		\subtype S T
	}{
		\convertible{S}{T}
	}
\trule{Con-IntBool}{
	}{
		\convertible{\Int}{\Bool}
	}
\trule{Con-Wrap}{
		\compatible S T
	}{
		\convertible{S}{\wrapty T}
	}
\end{trules}
\end{displayfigure}

In a traditional statically typed language, the type system fulfills two goals: 
\begin{enumerate}
\item
It detects certain errors at compile time.
\item
It guarantees what kinds of values are produced by certain expressions, which enables  run-time check elimination.
\end{enumerate}
The situation in ES4 is somewhat different, because of two reasons. First, in \t{standard} mode, we would like to eliminate run-time checks where possible, using a type-based analysis,
without reporting any compile-time type errors.
Second, \t{like} types weaken the guarantees provided by \t{strict} mode.
For these reasons, we actually present \emph{two} type systems.

For example, if a variable $x$ has type $\likety{\objty{f:\Int}}$,
then $\objget x f$  could actually return a value of any type.  
Nevertheless, 
$\objget x f$
would be expected to produce values of type $\Int$, and so 
the expression
$\objget{\objget x f} g$ would yield a compile-time type error.
Thus, we say that $\objget x f$ has type $\Int$, but this type is \emph{only} a statement of intent;
it does not guarantee what kinds of values are returned by that expression,
and so cannot be used for run-time check elimination.



The strict mode type system is based on a judgement $\judge{E}{e}{T}$,
stating that expression $e$ has type $T$ in environment $E$.
Note that the type $T$ \emph{only} indicates that $e$ is intended to produce values of type $T$; there are no guarantees here, due to the use of \t{like} types. Thus, the strict mode type system 
If $T$ is not a $\likety{}$ type, then $e$ will only produce values of that type.
If $T=\likety{T'}$, then the intent is that $e$ will only produce values of type $T'$,
but there is no guarantee. However, this intent can be still used to detect type errors at compile time.
%We sometimes use $k$ to range over $\likety{}^+$ or $\epsilon$.

Note that the judgement $\judge{E}{e}{T}$ means that $e$ is \emph{expected} to only produce values of type $T$,
and the purpose of the strict mode type system is only to heuristically detect errors at verification time.
The following section presents a type system with stronger guarantees that are sufficient for removing some run-time type checks.

If $\judge{E}{e}{T}$, then $T$ is not a \t{like} or \t{wrap} type (at least at the top level).

\begin{displayfigure}{th}{Type Rules for Strict Mode}
\label{fig:typerules}
\begin{trules}
\headtrule{$\judge{E}{t}{T}$}{\underline{T}ype rules\qquad\qquad\qquad\qquad\qquad\qquad\qquad\qquad\qquad\qquad\mbox{}}
\trule{T-Var}{
		(x:(\t{like}~|~\t{wrap})^*~T)\in E
	}{
		\judge{E}{x}{T}
	}
\trule{T-Const}{
	}{
		\judge{E}{c}{\fun{ty}(c)}
	}
\trule{T-Fun}{
		\judge{E,x:S}{e}{T'} \qquad
		\convertible{T'}{T} 
	}{
		\judge{E}{(\lam{x}{S}{T}{e})}{(\lamt{S}{T})}
	}
\trule{T-App1}{
		\judge{E}{t_1}{(\lamt{S}{T})}\qquad
		\judge{E}{t_2}{S'} \qquad
		\convertible{S'}{S} \\
	%	(k,k')\in\setz{(\likety{},\likety{\!\!}),(\wrapty{},\epsilon),(\epsilon,\epsilon)} \\
	}{
		\judge{E}{(\app{t_1}{t_2})}{T}
	}
\trule{T-App2}{
		\judge{E}{t_1}{\dynamic}\qquad
		\judge{E}{t_2}{S'}  
	}{
		\judge{E}{(\app{t_1}{t_2})}{\dynamic}
	}
\trule{T-As}{
		\judge{E}{t}{S}%\qquad 
		%\compatible{S}{T}
	}{
		\judge{E}{\cast{t}{T}}{T}
	}
\trule{T-Wrap}{
		\judge{E}{t}{S}%\qquad 
		%\compatible{S}{T}
	}{
		\judge{E}{\wrap{t}{T}}{T}
	}
%\trule{T-Let}{
%		\judge{E}{t_1}{S}\qquad
%		\judge{E,x:S}{t_2}{T} 
%	}{
%		\judge{E}{\letexp{x}{t_1}{t_2}}{T}
%	}
\trule{T-Alloc}{
		\judge{E}{t_i}{S_i} \qquad
		\convertible{S_i}{T_i} \qquad
		T=\objty{l_i:T_i^{i\in 1..n}}
	}{
		\judge{E}{(\obje{l_i=t_i^{i\in 1..n}}{T})}{T}
	}
\trule{T-Get1}{
		\judge{E}{e}{\objty{l:T,\dots}}  
		%k\in\setz{\likety{},\wrapty{},\epsilon}
	}{
		\judge{E}{\objget{e}{l}}{T}
	}
\trule{T-Get2}{
		\judge{E}{e}{\dynamic}
	}{
		\judge{E}{\objget{e}{l}}{\dynamic}
	}
\trule{T-Set1}{
		\judge{E}{e_1}{\objty{l:T,\dots}} \qquad
		\judge{E}{e_2}{S} \qquad
		\convertible{S}{T}  
	}{
		\judge{E}{\objset{e_1}{l}{e_2}}{S}
	}
\trule{T-Set2}{
		\judge{E}{e_1}{\dynamic} \qquad
		\judge{E}{e_2}{S}
	}{
		\judge{E}{\objset{e_1}{l}{e_2}}{S}
	}
\end{trules}
\end{displayfigure}

\comment{
\clearpage
\section{Sugar}
\[
\begin{array}{rcl}

\lam{x}{(\t{wrap}~{S})}{T}{e} &=&
	\lam{x}{(\likety{S})}{T}
	    {\lete{x}{S}{(\wrap{x}{S})}{e}}
	    \\~\\
\lam{x}{S}{(\t{wrap}~T)}{e} &=&
	\lam{x}{S}{T}{(\wrap{e}{T})}
	\\~\\
\t{wrap}~(\lam x S T e) &=&
	\lam{x}{(\t{wrap}~S)}Te
\end{array}
\]
}

%%%%%%%%%%%%%%%%%%%%%%%%%%%%%%%%%%%%%%%%%%%%%%%%%%%%%%%%%%%%%%%%%%%%%%%

\end{document}

